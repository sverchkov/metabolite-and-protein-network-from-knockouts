\documentclass[twocolumn]{article}

\usepackage{amsmath,amsfonts}

\title{Nested Effects Models: seeking an optimal approach}

\author{Yuriy Sverchkov}

\begin{document}
\maketitle

\section{Introduction}
We see a nested effects model as a way to express a nesting structure of sets of effects that are affected subject to an action.

\begin{table}
	\centering
	\begin{tabular}{l|ccc}
		      & $a$ & $b$ & $c$ \\ \hline
		$e_1$ &   1 &   0 &   0 \\
		$e_2$ &  -1 &   0 &   0 \\
		$e_3$ &   1 &   1 &   0 \\
		$e_4$ &  -1 &   1 &   0 \\
		$e_5$ &   0 &  -1 &   0 \\
		$e_6$ &   0 &  -1 &  -1 \\
		$e_7$ &   1 &   0 &  -1 \\
		$e_8$ &  -1 &   0 &  -1 \\
		$e_9$ &   1 &  -1 &  -1 \\
	\end{tabular}
	\caption{Toy example data}
\end{table}

Hence we define ``coverage'' of a set of actions as the number of effects affected by any action in the set, and not affected by subsets of actions outside the set.

\[
C( \{ a \} ) = 2
\]

\begin{table}
	\centering
	\begin{tabular}{l|rrr}
		$S$ & $C(S)$ & $C(S^+)$ & $C( S^- )$ \\ \hline
		$\emptyset$ & 0 & 9 & 0 \\
		$\{a\}$ & 2 & 7 & 2 \\
		$\{b\}$ & 1 & 5 & 1 \\
		$\{c\}$ & 0 & 4 & 0 \\
		$\{a,b\}$ & 2 & 3 & 5 \\
		$\{a,c\}$ & 2 & 3 & 4 \\
		$\{b,c\}$ & 1 & 2 & 2 \\
		$\{a,b,c\}$ & 1 & 1 & 9 \\
	\end{tabular}
	\caption{Coverage counts for toy example}
\end{table}

Let us use the shorthand $C( a )$ to represent $C( \{a\} )$, and let us use the shorthand $C( a^+ )$ to represent the number of effects affected by any action in $a$, i.e.:

\[
C( a^+ ) = C( \{a\}, \{a,b\}, \{a,c\} \ldots ) = 7
\]

The ideal NEM is a DAG such that (let $A(n)$ be the ancestry of $n$ in the DAG)
\[
\max \sum_n C( A( n ) )
\]

\section{A greedy search}

Constructing the DAG one node at a time is tough because being a DAG, we must consider many possible combinations of parents.

\section{Another approach}
Consider the case where we must arrange the actions in a single thread $a \rightarrow b \rightarrow \ldots$.
The node at the top must necessarily be the one that maximizes $C(n^+)$, and each node subsequent should maximize $C(A(n)^+)$.

\paragraph{Proof} To be added.

We can next consider constructing for each node $n$ the best strand rooted at $n$.
If a strand rooted at $n$ scores lower or equal to some $C(\{n,m\}^+)$ $n$ can be eliminated from consideration as a root node.

\end{document}